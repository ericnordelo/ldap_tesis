\begin{opinion}
La Dirección de Informatización de la Universidad de la Habana, también conocida como DDI,  es la principal encargada de informatizar los procesos en nuestra casa de altos estudios. Para poder ofrecer servicios orientados al personal de la universidad fue necesario idear un sistema de alta disponibilidad que permite la creación y autenticación de cuentas para sus usuarios. Dicho sistema, al cual se nombró Directorio Único, surgió y creció en condiciones irregulares que hoy día hacen que su desempeño se desvie del objetivo original con el que fue concebido. 

En dicho contexto, surge la tesis \textit{Sistema de extracción y publicación de información relacionada con los Recursos Humanos de la Universidad de la Habana}. Este trabajo realiza un estudio de como extraer la información del personal universitario de sus fuentes originales sin la necesidad de utilizar un sistema intermediario como sería el actual Directorio Único de la UH. También, propone interfaces alternativas a las actuales para la interación y presentación con dichos datos.

Para cumplir con los objetivos de este trabajo el estudiante Eric Nordelo Galeano tuvo que investigar y enteder la problemática del fiujo de información dentro de la universidad y proponer una solución acorde a los objtivos deaseados. 

Aunque a este proyecto aun dista de ser un sistema capaz de entrar en producción cumple con los objetivos inicialmente trazados.Considero que el resultado de esta tesis es suficiente para demostrar que Eric pueda desempeñarse como Científico de la computación.	
\end{opinion} 