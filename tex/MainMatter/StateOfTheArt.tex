\chapter{Estado del arte}

\section{LDAP}
\subsection{¿Qué es el protocolo LDAP?}
LDAP (Lightweight Directory Access Protocol o Protocolo de Acceso a Directorios Ligero) es un protocolo perteneciente a la capa de 
aplicaciones \cite{ ldap1}, tanto para servidores como para clientes. Es abierto y 
multiplataforma. Está pensado para la implementación de servicios de directorio, como son: 
\textbf{IBM Security Directory Server}, \textbf{Active Directory}, \textbf{Oracle Internet 
Directory} y \textbf{OpenLDAP}; facilitando el acceso rápido a la información almacenada. 
Presenta una estructura arbórea, la cual organiza la información en ramas y permite realizar 
búsquedas de manera eficiente, debido a que la cardinalidad de las posibles repuestas se 
reduce a medida que se avanza por cualquiera de estas ramas.

La selecci\'on de la implementación utilizada es objetivo de la tesis que se encarga del servidor LDAP, en esta se utiliza el este protocolo para comunicarse con este servidor, por eso se menciona qu\'e es este protocolo. En los anexos se puede encontrar información más espec\'ifica sobre el mismo, pero no es objetivo principal de esta investigación.

\section{Docker}

\subsection{¿Qué es Docker?}
Docker es una plataforma de software libre que permite crear, desplegar y administrar 
contenedores de aplicaciones virtualizadas, sobre un mismo sistema operativo. Facilita la 
integración de distintas aplicaciones independientemente del sistema que estas necesiten para
funcionar. Hasta el momento de escritura de la presente tesis, Docker es capaz de crear 
contenedores basados en cualquier sistema Unix-Like \footnote{ Sistemas que utilizan el kernel de
Linux} y en Windows. Todavía no existe forma de hacerlo basado en macOS \footnote{ 
\url{https://www.apple.com/la/macos/what-is/}}.

\subsection{Ventajas de utilizar Docker}

Docker es sumamente útil en el proceso de despliegue de aplicaciones. A través de esta tecnología
es posible planificar todo el proceso de despliegue de un sistema, plasmarlo en una secuencia de
pasos lógicos y luego simplemente ejecutarlo. De esta manera se reduce el tiempo necesario para
este proceso evitando errores frecuentes y acciones repetitivas.

En el caso espec\'ifico que concierne a esta tesis, es posible utilizar Docker para automatizar todo el proceso de despliegue en producci\'on de la solución que se propone.

\subsection{¿Qui\'en utiliza Docker \cite{webinars}?}

Muchas grandes empresas se han apoyado en Docker para automatizar el desarrollo de software. Como ejemplo encontramos algunas muy conocidas mundialmente como son Spotify, Uber, eBay o PayPal.

Tomando como caso específico a eBay, ésta ha centrado sus esfuerzos en la implementación de Docker en su proceso de integración continua, estandarizando sus instalaciones mediante despliegues en una red distribuida de servidores que se ejecutan y trabajan en grupo. Esto permite a los desarrolladores o testers (encargados de realizar pruebas de software) sólo tengan que montar contenedores sobre sus sistemas para poder ponerse a trabajar, con toda la configuración, herramientas o servicios precargados, lo que agiliza enormemente los tiempos de desarrollo o testeo de versiones.

En general estas empresas (y muchas otras) vieron en Docker la respuesta a sus preguntas de cómo poder obtener mayor rendimiento invirtiendo menos horas de supervisión y desarrollo. Creando una imagen que contenga el sistema, las configuraciones y servicios listos para ser desplegados; se ahorraron multitud de horas de gestión de sistemas, reparación, recuperación y reinstalación de éstos.

Gracias a los contenedores las pruebas se realizan en entornos “cerrados” por lo que si se diese algún error la mayor complicación que se encontrarán será la de volver a desplegar el contenedor de nuevo.

\subsection{¿Qué es un contenedor de docker?}

Como parte fundamental de Docker se encuentran los contenedores. Estos, según los propios
desarrolladores, \textit{son unidades de software estandarizadas} \footnote{
\url{https://www.docker.com/resources/what-container}}. Se encargan de empaquetar
el código y las dependencias de la aplicación de manera que pueda ser ejecutado en cualquier 
entorno capaz de ejecutar Docker. Para ejecutar un contenedor, se necesita una imagen de Docker 
base. Una imagen de Docker no es más que un paquete que contiene todas las dependencias y 
configuraciones necesarias para ejecutar un contenedor a partir de esta. Los contenedores aíslan
al software del entorno en que son ejecutados y aseguran el correcto funcionamiento 
independientemente de las posibles diferencias entre un entorno u otro. Sin embargo, dicho
aislamiento no impide la comunicación entre varios contenedores. Existen mecanismos que permiten
interactuar a través de servicios de red. Incluso es posible crear una red privada que incluya
ciertos contenedores. Con esta facilidad, se puede diseñar soluciones estructuradas en módulos
atómicos que se integren como sistema.

\subsection{Docker Compose}

Compose es una herramienta para definir y ejecutar aplicaciones en Docker con contenedores múltiples. Con Compose, se utiliza un archivo YAML para configurar los servicios de la aplicación. Luego, con un solo comando, crea e inicia todos los servicios desde la configuración \cite{compose}. Un ejemplo de archivo de configuración para Docker Compose es el siguiente:

\begin{tabbing}
	vers\=ion: '3' \\

	services: \\
	\> memc\=ached:  \\
	\> \>	build: \\
	\> \>	context: . \\
	\> \>	dockerfile: ./memcached-dockerfile \\
	\> \>	image: memcached \\
	\> \>	container\_name: memcached\_server \\
	\> \>	expose: \\
	\> \>	- 11211 \\
	\> \>	ports: \\
	\> \>	- "11211:11211" \\
	\> \>	restart: always \\
	\> \>	command: memcached -u root \\
	\> db  \=: \\
	\>\>	image: "mcr.microsoft.com/mssql/server" \\
	\>\>	container\_name: sqlserver\_db \\
	\>\>	environment: \\
	\>\>	SA\_PASSWORD: "P@ssw0rd" \\
	\>\>	ACCEPT\_EULA: "Y" \\
	\> api \=: \\
	\>\>	build: . \\
	\>\>	image: flask\_restful \\
	\>\>	container\_name: ldap\_api \\
	\>\>	expose: \\
	\>\>	- 5000 \\
	\>\>	ports: \\
	\>\>	- "5000:5000" \\
	\>\>	volumes: \\
	\>\>	- .:/api \\
	\>\>	restart: always \\
	\>\>	working\_dir: /api \\
	\>\>	command: python run.py \\
	\>\>	depends\_on: \\
	\>\>	- memcached \\
	\>\>	- db \\
	\> ui  \=: \\
	\>\>	build: \\
	\>\>	context: . \\
	\>\>	dockerfile: ./ui-dockerfile \\
	\>\>	image: react\_ui \\
	\>\>	container\_name: ldap\_ui \\
	\>\>	ports: \\
	\>\>	- "3000:3000" \\
	\>\>	working\_dir: /ui \\
	\>\>	volumes: \\
	\>\>	- ./UI:/ui \\
	\>\>	restart: always \\
	\>\>	command: bash -c "npm install \&\& npm start" \\
		
	\> netw\=orks: \\ 
	\>\> defa\=ult: \\
	\>\>\>	exte\=rnal: \\
	\>\>\>\>	name: tesis \\
	% If \= it's raining \\
	% \> then \= put on boots,\\
	% \> \> take hat;\\
	% \> or \> smile. \\
	% Leave house.
\end{tabbing}

\subsection{Casos de uso comunes de Docker Compose}

\subsubsection{Entornos de desarrollo}

Cuando se desarrolla software, la capacidad de ejecutar una aplicación en un entorno aislado e interactuar con ella es crucial. La herramienta Compose se puede utilizar para crear el entorno e interactuar con él.

El archivo de configuración proporciona una forma de documentar y configurar todas las dependencias de la aplicación (bases de datos, colas, cachés, API de servicios web, etc.). Con esta herramienta, se puede crear e iniciar uno o más contenedores para cada dependencia con un solo comando (docker-compose up).

Juntas, estas características proporcionan una manera conveniente para que los desarrolladores comiencen un proyecto. Compose puede reducir una "guía de inicio para desarrolladores" de varias páginas a un solo archivo Compose y algunos comandos.


\subsubsection{Entornos de pruebas automatizadas}

Una parte importante de cualquier proceso de implementación continua o integración continua es el conjunto de pruebas automatizadas. Las pruebas automatizadas de extremo a extremo requieren un entorno en el que ejecutarse. Compose proporciona una forma conveniente de crear y destruir entornos de prueba aislados. Al definir el entorno completo en un archivo de configuración, se puede crear y destruir estos entornos con solo unos pocos comandos:

docker-compose up -d

./run\_tests

docker-compose down



\section{API}

\subsection{¿Qué es una API?}

Una API es un conjunto de funciones y procedimientos que cumplen una o muchas funciones con el fin de ser utilizadas por otro software. Las siglas API vienen del inglés Application Programming Interface. En español sería Interfaz de Programación de Aplicaciones. Es una especificación formal sobre cómo un módulo de un software se comunica o interactúa con otro.
\footnote{\url{https://hipertextual.com/archivo/2014/05/que-es-api/}}

\subsection{Ventajas de implementar una API}

Las API simplifican en gran medida el trabajo de un creador de programas, ya que no tiene que escribir códigos desde cero. Estas permiten al programador utilizar funciones predefinidas para interactuar con el sistema operativo o con otro programa. También reducen los costes de mantenimiento, al organizar de forma estructurada los sistemas y procesos internos, lo cual permite la integración de nuevos proyectos de una manera uniforme. También agilizan los procesos de transformaci\'on, ya que mientras la interfaz sea la misma, los procesos internos pueden cambiar tanto como sea necesario, sin la necesidad de modificar los servicios que consumen una API espec\'ifica.

En el caso de la solución que propone esta tesis, tener una API como intermediaria entre el servidor LDAP, las demás fuentes de información (SIGENU y Bases de Datos de trabajadores) y los mecanismos de gestión de información (interfaces visuales para manejo de externos, cuotas, etc...), permite manipular los diversos componentes del Directorio Único utilizando un lenguaje común muy estandarizado (protocolo HTTP en este caso). Esto permite a los distintos clientes consumidores de la información de directorio abstraerse del protocolo LDAP (menos estandarizado). Además permite que al modificarse la estructura del Directorio, los clientes no tengan que ser modificados, ya que al modificar solamente la API como capa intermedia podemos cambiar la forma de obtener información de las fuentes mientras mantenemos la estructura a la hora de publicar la información. 

\subsection{Tecnologías para implementación más utilizadas}

Para la implementación de una API como servicio web existen gran cantidad de \textit{frameworks} \footnote{\url{https://www.ecured.cu/Framework}} sobre una gran variedad de lenguajes de programación. Entre los más utilizados tenemos:

\subsubsection{ASP.NET Core}

ASP.NET Core es un framework web de código abierto implementado por Microsoft, para desarrollar aplicaciones web dinámicas utlizando C Sharp o Visual Basic como lenguaje de programación. 

Este framework es un rediseño de ASP.NET 4.x, para crear aplicaciones web. Aquí tenemos los beneficios que brinda según la documentación oficial de Microsoft \cite{netcore}:

\begin{itemize}
    \item Una historia unificada para desarrollar interfaces web y APIs.
    \item Arquitectura diseñada para pruebas.
    \item Las páginas Razor facilitan y hacen más productiva la codificación enfocada en páginas web.
    \item Blazor te permite utilizar C Sharp junto a Javascript en el navegador.
    \item Permite desarrollar y ejecutar aplicaciones tanto en Windows, como en Linux y macOS.
    \item Es de código abierto y basado en la comunidad.
    \item Permite una fácil integración de frameworks del lado del cliente modernos.
    \item Se adapta fácilmente a distintos flujos de desarrollo modernos.
    \item Aplicaciones listas para desplegar en la nube.
    \item Inyección de dependencias integrada.
    \item Un flujo de peticiones HTTP ligero, eficiente y modular.
    \item Permite ser desplegado utilizando IIS, Nginx, Apache, Docker.
    \item Posee varias herramientas que simplifican el desarrollo web moderno.
\end{itemize}

\subsubsection{ASP.NET MVC}

Este es otro framework desarrolladores por Microsoft basado en .NET, con la diferencia fundamental de que implementa el patr\'on Modelo Vista Controlador. Es de código abierto también, exceptuando ASP.NET Web Forms que es un software propietario.

El patrón Modelo Vista Controlador (MVC) es un estilo de arquitectura de software que separa los datos de una aplicación, la interfaz de usuario, y la lógica de control en tres componentes distintos: el Modelo que contiene una representación de los datos que maneja el sistema, su lógica de negocio, y sus mecanismos de persistencia; la Vista, o interfaz de usuario, que compone la información que se envía al cliente y los mecanismos interacción con éste; el Controlador, que actúa como intermediario entre el Modelo y la Vista, gestionando el flujo de información entre ellos y las transformaciones para adaptar los datos a las necesidades de cada uno \cite{mvc}.

Entre las ventajas de la utilizaci\'on de este framework est\'an:

\begin{itemize}
    \item Posibilidad de dividir la lógica de negocio del diseño, haciendo el proyecto más escalable.
    \item Facilita el uso de URL amigables, importantes para el SEO (posicionamiento web), la mayoría de frameworks MVC lo controlan.
    \item Puedes utilizar abstracción de datos, como lo hacen también otros frameworks como Ruby on Rails, mediante el uso de ORMs (Object-Relational Mapping), permitiendo comunicarte con la base de datos de la aplicaci\'on de una forma orientada a objetos y no mediante SQL.
\end{itemize}

\subsubsection{Ruby on Rails}

Este es un framework web de código abierto con una de las comunidades de desarrolladores más grandes. Esta implementado sobre el lenguage Ruby. Utiliza también el patrón Modelo Vista Controlador (MVC), con las ventajas que vimos anteriormente. Es un sistema web integrado que incorpora por defecto muchas de las herramientas necesarias para escalar un sitio web de manera rápida y eficiente. Como Ruby es un lenguage dinámico \cite{dynamic} el desarrollo es más veloz debido a que no se necesita recompilar el código para probar cambios. Entre las principales ventajas de elegir Ruby on Rails tenemos:

\begin{itemize}
    \item Utiliza el patrón MVC.
    \item Permite desplegar de forma separada contenido estático y dinámico.
    \item Dispone de un sistema de caché integrada que se puede utilizar a nivel de página, o en cualquier fragmento según sea necesario en cada caso.
    \item ORM Integrado. El famoso ORM ActiveRecord permite lanzar consultas a la DB de forma extremadamente fácil y sin necesidad de escribir nada de SQL.
    \item Sistema de testing integrado. Cuando hay la necesidad de implementar algoritmos complejos es mejor utilizar TDD para tener una buena cobertura que les dé validez y estabilidad.
    \item Ecosistema de librerías maduro. Existen librerías para casi cualquier necesidad (existe una gran cantidad de módulos denominados gemas, que agregan casi cualquier funcionalidad a la aplicación).
\end{itemize}

\subsubsection{Django}

Django es un framework gratuito y de código abierto basado en Python, que sigue el patrón Modelo Plantilla Vista. Lo mantiene Django Software Foundation, una organización independiente establecida como una organización sin fines de lucro. El objetivo principal de Django es facilitar la creación de sitios web complejos basados en bases de datos.

En este patrón MTV, la M significa Model (Modelo), la capa de acceso a la base de datos. Esta capa contiene toda la información sobre los datos: cómo acceder a estos, cómo validarlos, cuál es el comportamiento que tiene, y las relaciones entre los datos.
T significa Template (Plantilla), la capa de presentación. Esta capa contiene las decisiones relacionadas a la presentación: como algunas cosas son mostradas sobre una página web o otro tipo de documento.
V significa View (Vista), la capa de la lógica de negocios. Esta capa contiene la lógica que accede al modelo y la delega a la plantilla apropiada: se puede pensar en esto como un puente entre el modelos y las plantillas \cite{mtv}. 

Es un patrón muy similar a MVC, excepto que la C (Controlador) es gestionada en el mismo framework (a diferencia de otros como ASP.NET MVC o Ruby on Rails). En Django, las vistas tienen más relación con los controladores de MVC, y las plantillas con las vistas. Resumiendo las ventajas de usar Django encontramos:

\begin{itemize}
    \item Utiliza el patrón MTV (muy similar a MVC).
    \item Está escrito en Python, lenguaje con una gran cantidad de librerías para resolver diversos tipos de problemas.
    \item Es el único framework que por defecto viene con un sistema de administración activo, listo para ser utilizado sin ningún tipo de configuración.
    \item ORM Integrado.
\end{itemize}

\subsubsection{Flask}

Flask es un microframework web escrito en Python. Se clasifica como un microframework debido a que no requiere de herramientas ni librerías particulares. No tiene una capa de abstracción de base de datos ni sigue un patrón de desarrollo específico. Las funciones específicas como la autenticación y el manejo de base de datos son brindadas por software de terceros. Entre las principales ventajas de este framework minimalista tenemos que es muy eficiente, ya que te permite seleccionar que componentes utilizar específicos a la plataforma que estés desarrollando. Es muy utilizada para desarrollar APIs debido a la facilidad con que se pueden desarrollar las mismas (muy pocas líneas de código), y debido a que está montado sobre python permite utilizar todas las libreríascon las que cuenta este poderoso lenguaje de programación. Tiene una comunidad de desarrolladores activa y una buena cantidad de módulos para agregar distintas funcionalidades a los servicios web.

\section{python-ldap}

\subsection{¿Qué es python-ldap?}

python-ldap es una librería del lenguaje de programación Python, la cual se encarga de proveer una API orientada a objetos para acceder a servidores de directorios basados en el protocolo LDAP desde los programas de Python.


\section{Interfaz web}

Por último, existen diferentes tecnologías para implementar la interfaz web, que facilitará la interacción de los usuarios, tanto familiarizados con la programación como ajenos a la misma (secretarias y personal de Recursos Humanos), con el API e indirectamente con el servidor LDAP y las fuentes de información. Como para mantener modularizada la solución, el API y la interfaz deben ser dos programas separados, en este caso es muy factible utilizar una SPA (Single Page Application), que funcione como una aplicación independiente. Algunas de las tecnologías mas usadas para desarrollar SPAs actualmente son:

\subsubsection{Angular}

Este es un framework monolítico de código abierto desarrollado por Google, muy poderoso por la cantidad de funcionalidades internas que provee pero con una curva de aprendizaje excesiva. 

\subsubsection{React}

Este es la alternativa de Facebook, ampliamente utilizada al igual que Angular, también es de código abierto y aunque no es un framework en sí (es una librería para renderizar vistas), unida con otras tecnologías como redux, se convierte en una herramienta poderosa para dessarrollar SPAs. Es muy eficiente y su curva de aprendizaje es más suave que la de Angular.

\subsubsection{Vuejs}

Este es la alternativa más joven de las tres, la cuál aún con la comunidad más pequeña a obtenido una gran cantidad de estrellas en github.com debido a que tiene la curva de aprendizaje más suave, y a las comodidades que brinda por su semántica sensilla a los desarrolladores.