\chapter{Implementación del sistema}
\section{Consideraciones previas}

La implementación del servidor LDAP es parte del desarrollo otra tesis que complementa a esta en la creación de la solución final. En esta se implementó lo concerniente al API y la interfaz web.

\section{API}

La API se desarrolló utlizando el framework Flask de Python, debido a la eficiencia de este framework minimalista, y al poder de este lenguaje de programación en cuanto a soluciones ya implementadas (librerías). Consiste en un conjunto de recursos creados utilizando la librería flask-restful a los cuales se accede utilizando el protocolo HTTP (los recursos se publican como URLs).  Utilizando la librería python-ldap, el API se conecta y se comunica con el servidor LDAP. 

\subsection{Listado de recursos}

\subsubsection{/login}

Este recurso es utilizado para autenticar a los usuarios en el API.

\textbf{Método:} POST

\textbf{Argumentos:} [username, password]

\textbf{Objeto respuesta:} []



\subsubsection{/logout}

Este recurso es para terminar la sesión de un usuario autenticado en el API.

\textbf{Método:} POST

\textbf{Argumentos:} []

\textbf{Objeto respuesta:} []

\subsubsection{/usuarios}

Este recurso es para obtener la información pertinente un conjunto de personas pertenecientes a la universidad en base a parámetros de búsqueda.

\textbf{Método:} GET

\textbf{Argumentos:} [filtros]

\textbf{Objeto respuesta:} [{usuarios: [...]}]

\subsubsection{/trabajadores}

Este recurso es accesible mediante 3 métodos HTTP, para realizar 3 acciones distintas. Método GET para obtener la información pertinente a un conjunto de trabajadores pertenecientes a la universidad en base a parámetros de búsqueda. Método POST para agregar un trabajador al directorio a partir de su Carnet de Identidad. Método PATCH para actualizar los trabajadores en el directorio a partir de las Bases de Datos de Recursos Humanos (cuando se hace mención al directorio nos referimos al servidor LDAP que almacena la información).

\textbf{Métodos:} GET, POST, PATCH

\textbf{Argumentos:} GET: [filtros], POST: [ci], PATCH: []

\textbf{Objeto respuesta:} GET: [{'workers': [...]}], POST: [], PATCH: []

\subsubsection{/estudiantes}

Este recurso es accesible también mediante 3 métodos HTTP, para realizar 3 acciones distintas. Método GET para obtener la información pertinente a un conjunto de estudiantes de la universidad en base a parámetros de búsqueda. Método POST para agregar un estudiante al directorio a partir de su Carnet de Identidad. Método PATCH para actualizar los estudiantes en el directorio a partir de SIGENU (cuando se hace mención al directorio nos referimos al servidor LDAP que almacena la información).

\textbf{Métodos:} GET, POST, PATCH

\textbf{Argumentos:} GET: [filtros], POST: [ci], PATCH: []

\textbf{Objeto respuesta:} GET: [{'students': [...]}], POST: [], PATCH: []

\subsubsection{/externos}

Este recurso es accesible mediante 2 métodos HTTP, para realizar 2 acciones distintas. Método GET para obtener la información pertinente a un conjunto de externos de la universidad en base a parámetros de búsqueda. Método POST para agregar un externo al directorio (cuando se hace mención al directorio nos referimos al servidor LDAP que almacena la información).

\textbf{Métodos:} GET, POST

\textbf{Argumentos:} GET: [filtros], POST: [nombre, ci, cuota, ...]

\textbf{Objeto respuesta:} GET: [{'externs': [...]}], POST: []

\subsubsection{/p/preguntasdeseguridad}



\subsubsection{/p/cambiar}

\section{Interfaz web}
\section{Desarrollo de la solución}\label{cap:Solution}

\begin{figure}
\end{figure}




  
