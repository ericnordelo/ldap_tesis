\begin{abstract}
    En este trabajo se desarrolla una solución informática al problema de acceso a la información de Recursos Humanos (relacionada con todo el personal docente y no docente) que presenta la Universidad de La Habana. La entidad actualmente cuenta con un sistema que se encarga de gestionar esta información (Directorio Único), el cual presenta ciertas deficiencias como la pérdida de códigos fuentes y la descentralización desorganizada de los servicios que lo componen. El objetivo es sustituir este sistema por uno más robusto, estable y eficiente, que sea fácilmente modificable y sostenible. Se propone un sistema basado en el protocolo LDAP y una interfaz que permita la comunicación con el mismo de manera fácil, así como la unificación de los servicios actualmente dispersos en uno solo (o la menor cantidad posible). Para esto la propuesta se compone de una API y una interfaz web, que se comuniquen con un servidor LDAP que se encargue de almacenar la información (el despliegue de este servidor es objetivo de otra tesis que se desarrolla en pararelo). Como culminación del trabajo se espera la adopción de este sistema en la Universidad de La Habana.
    
    \textbf{Palabras clave}: Directorio Único, LDAP, API, interfaz web.
\end{abstract} 