\begin{conclusions}
    En la presente tesis se desarrolló un sistema que da solución al problema que presenta la utilización del actual Directorio Único de la Universidad de La Habana, manteniendo las ventajas que el mismo brinda para tener acceso a la información relacionada con Recursos Humanos (aclarar que con Recursos Humanos se hace referencia a todo el personal de las unidades presupuestadas que pertenecen a la universidad, incluyendo a los externos y a los estudiantes), y eliminando las carencias más importantes del mismo como la descentralización desorganizada de los servicios que lo componen, la pérdida de código fuente, la heterogeneidad de las tecnologías utilizadas en su composición y la inestabilidad.

    Como objetivos específicos se trabajó la extracción de la información de las distintas fuentes de datos (SIGENU y las bases de datos de ASSETS); la adición de datos a partir de las fuentes al servidor LDAP ya implementado; la creación de una capa intermedia para facilitar la comunicación y presentación de los datos de este servidor; y la implementación de una interfaz para consultar y gestionar la información que en él se almacena.   

    Para el primer objetivo mencionado (extracción de información de las fuentes de datos), después de estudiada la forma en que se publican los datos en cada fuente en particular, se implementaron dos módulos independientes en Python (uno por cada fuente). El primero se encarga de obtener una salva de una de las Bases de Datos de Assets (que se publica en un ftp en la red de la universidad), restaurarla en un servidor MS SQL Server desplegado utilizando la herramienta Docker, y obtener la información de esta base de datos. El segundo se encarga de extraer la información del sistema SIGENU, utilizando servicios que publica este sistema a través del protocolo SOAP.

    Para el segundo objetivo mencionado (adición de la información al servidor LDAP), se utilizan también estos dos módulos mencionados anteriormente. Ambos módulos después de extraer la información como se explicó en el párrafo anterior, se encargan de generar archivos LDIF con las entradas listas para modificar los datos almacenados en el servidor LDAP a partir de esta información. Luego utilizan este archivo LDIF y la librería python-ldap para realizar la modificación.

    Para el tercer objetivo (creación de una capa intermedia para facilitar la comunicación y presentación de los datos), se implementó una API, la cual posee un mecanismo de autenticación y un sistema de roles para controlar como se publican y se gestionan los datos almacenados en el servidor. Esta API fue diseñada para ser fácilmente extensible, para permitir cubrir las necesidades que vayan surgiendo a medidas que aparezcan nuevos servicios en la red o se modifiquen los ya existentes (que dependen de esta información en el servidor LDAP). Está implementada en Python por lo cual es fácil de integrar directamente con los módulos que cubren los dos primeros objetivos. Esta API se configura y despliegua utilizando la herramienta Docker, lo cual hace muy sencillo estos dos procesos.

    Para el objetivo número cuatro (implementación de una interfaz para consultar y gestionar la información), se implementó una SPA utlizando la tecnología React. Esta interfaz se comunica con el API para obtener la información y no directamente con el servidor LDAP, o sea, es uno más de los servicios que consumen el API para tratar con la información relacionada con Recursos Humanos.                                                    

    Gracias al uso de la tecnología Docker el despliegue de toda la solución requiere una configuración manual mínima, así como ninguna dependencia en la computadora final en la que se publique el sistema como servicio (exceptuando Docker). Además, el uso de lenguajes dinámicos (como lo son Python y Javascript), disminuye el riesgo de la pérdida del código fuente, ya que la aplicación en producción no es un archivo binario ilegible, sino el código en si.

    
\end{conclusions} 