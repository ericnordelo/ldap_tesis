\begin{abstract}
    En este trabajo se desarrolla una solución informática al problema de acceso a la información de Recursos Humanos 
    \todo[inline]{Aclarar que cuando se dice recursos humanos se entiende todo el personal docente y no docente de la universidad para evitar malentendidos con la definicion de RH de la UH.}
    que presenta la Universidad de La Habana. La entidad actualmente cuenta con un sistema poco estable e insostenible por diversos factores (Directorio Único)
    \todo[inline]{Como regla evitemos hablar de mala forma de directorio Unico.}
    , entre los que están la pérdida de códigos fuentes y la descentralización desorganizada de los servicios que lo componen. El objetivo es sustituir este sistema por uno más robusto, estable y eficiente, que sea fácilmente modificable y sostenible. Se propone un sistema basado en el protocolo LDAP y una interfaz que permita la comunicación con el mismo de manera fácil, así como la unificación de los servicios actualmente dispersos en uno solo (o la menor cantidad posible). Para esto la propuesta se compone un servidor LDAP, una API y una interfaz web. Como culminación del trabajo se espera la adopción de este sistema en la Universidad de La Habana.
    
    \todo[inline]{La propuesta no incluye un servidor LDAP. }
    
    \textbf{Palabras clave}: Directorio Único, LDAP, API, interfaz web.
\end{abstract} 