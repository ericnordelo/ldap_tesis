\chapter{Implementación del sistema}
\section{Consideraciones previas}

La implementación del servidor LDAP es parte del desarrollo de otra tesis que complementa a esta en la creación de la solución final. En este trabajo se implementó lo concerniente al API y la interfaz web, incluyendo la generación de los archivos LDIF utilizados para popular y modificar de manera periódica el servidor LDAP mediante dos módulos de Python independientes al API (aunque esta los utiliza); uno para la extracción de la información pertinente a los estudiantes y otro para la pertinente a los trabajadores. 
Además el API brinda una bolsa de servicios a los que se pueden asociar otros sistemas que necesitan consumir la información almacenada en el servidor (como por ejemplo la cuota de internet de un estudiante).

\section{Extracción de la información de estudiantes y trabajadores}

Como mencionamos antes, de la extracción de la información se encargan dos módulos independientes desarrollados en Python, los cuáles consumen la información de Recursos Humanos de los destinos correspondientes y generan archivos LDIF que utiliza el servidor LDAP para modificar los datos almacenados. Estos módulos son:

\subsection{Módulo para generar los LDIF de los estudiantes}

Este módulo se llama sigenu-client. Consume utilizando el protocolo SOAP los recursos que ofrece el sistema SIGENU para consultar información relacionada a los estudiantes de la universidad. Específicamente consume dos recursos: Faculties y getStudentsByFaculty. El primero se utiliza para obtener todas las facultades miembros de la universidad, y a partir de estas se utiliza el segundo para obtener todos los estudiantes matriculados en cada una de estas.

Está compuesto por dos archivos: el init, donde está el código fuente del módulo, y un archivo de configuración llamado config.yml, donde se define el esquema que utiliza el código para la generación del archivo LDIF final. Este esquema es una lista de pares ordenados donde el primero es el nombre del atributo a generar en el LDIF, y el segundo el nombre del campo a consultar después de extraer la información de SIGENU.

El código fuente tiene dos dependencias: El módulo yaml para la lectura del archivo de configuración; y el módulo zeep, que se encarga de comunicarse mediante el protocolo SOAP con los recursos de SIGENU. En el mismo se declara una clase SigenuClient, que recibe la dirección del archivo de configuración y el uid number a partir del cual empezar a generar las entradas en el LDIF. Este segundo parámetro es necesario ya que el uid tiene que ser único en el servidor LDAP, así que debe obtenerse del mismo el más grande e ir generando de forma incremental los posteriores. El método de instancia generate\_ldif se encarga de generar el archivo final.   

Al desarrollar este módulo se tuvo que lidiar con problemas inherentes al sistema SIGENU, como datos en mal estado (caracteres no alfanuméricos) y duplicación de cuentas. La política para el primer caso es no almacenar los estudiantes con datos en mal estado en el servidor LDAP, para que soliciten la apertura de la cuenta nuevamente y quede en correcto estado. La política para el segundo caso es no almacenar ninguna de las cuentas para que también sea necesario la creación de una nueva en correcto estado.



La generación del archivo LDIF final queda estructurada de la forma que puede apreciarse en el siguiente ejemplo:

\begin{enumerate}
	\item {\bf becado :} Externo
	\item {\bf carrera:} Ciencias Alimentarias
	\item {\bf ci:} 00071068259
	\item {\bf  ciudadania\footnote{sin acento por la codificación del archivo generado, para facilitar la integración con el servidor LDAP}:} Cuba
	\item {\bf cn:} Jeidis Elisa 
	\item {\bf correo:} None
	\item {\bf edad:} 19
	\item {\bf facultad:} None
	\item {\bf grado:} 1
	\item {\bf grupo:} 2
	\item {\bf lugardenacimiento:} None
	\item {\bf municipio:} Playa
	\item {\bf pais:} Cuba
	\item {\bf provincia:} La Habana
	\item {\bf raza:} B
	\item {\bf sexo:} Femenino
	\item {\bf tipodecurso:} Curso Regular Diurno
	\item {\bf idsigenu:} 2b2f70cf:1659152c466:603
	\item {\bf dn:} Estudiante
	\item {\bf idfacultad:} 223.0.06816\_12
	\item {\bf cuotadeinternet:} 0
	\item {\bf pcc:} False
	\item {\bf ujc:} False
	\item {\bf esBaja:} True
	\item {\bf objectclass:} Estudiante
	\item {\bf objectclass:} posixAccount
	\item {\bf objectclass:} shadowAccount
	\item {\bf uidNumber:} 5000
	\item {\bf gidNumber:} 10000
	\item {\bf homeDirectory:} /home/jeidis.elisa
	\item {\bf uid:} jeidis.elisa
\end{enumerate}

Después de generado el archivo LDIF, este es utilizado para poblar o modificar el servidor LDAP utilizando un parser de LDIF que brinda la biblioteca python-ldap.

\subsection{Módulo para generar los LDIF de los trabajadores}


Este módulo se llama ldif-from-database. Se encarga de obtener la base de datos de Recursos Humanos (Assets), extraer la información pertinente a los trabajadores, y generar el archivo LDIF utilizado por el servidor LDAP.

Es muy similar al módulo sigenu-client en su estructura. También está compuesto de un archivo init con el código fuente, y un archi de configuración nombrado config.yml, en el cual se especifica la dirección de la base de datos fuente, la dirección del servidor MS SQL Server en que se monta la misma, para la adecuada extracción de la información, y el esquema de los datos a consumir, que no es más que una lista de pares ordenados donde el primero representa el nombre del atributo a generar en el archivo LDIF final, y el segundo el número de la columna en la que aparece.

El código fuente tiene como dependencias el módulo yaml para la lectura del archivo de configuración, y un módulo implementado como parte de la solución de esta tesis llamado sqlserver\_manager. Este último módulo se encarga de obtener la base de datos de ASSETS del ftp en el cual se publica, utilizando el módulo de Python ftplib, para luego restaurarla utilizando el módulo pyodbc en el servidor sqlserver del cual posteriormente se van a consultar los datos (no pueden consultarse directamente del sistema fuente, debido a la imposibilidad de crear un usuario con permisos de solo lectura en el sistema propietario que se instaló).

La generación del archivo LDIF final queda estructurada de la forma que puede apreciarse en el siguiente ejemplo:

\begin{enumerate}
	\item {\bf ci:} 00010868492
	\item {\bf cn:} DAIMIS DE LA CARIDAD
	\item {\bf area:} COMUNICACION
	\item {\bf cargo:} SECRETARIA
	\item {\bf sexo:} F
	\item {\bf sn:} HECHEVARRIA GUERRA
	\item {\bf dn:} 2
	\item {\bf objectclass:} Trabajador
	\item {\bf objectclass:} posixAccount
	\item {\bf objectclass:} shadowAccount
	\item {\bf uidNumber:} 5000
	\item {\bf gidNumber:} 10000
	\item {\bf homeDirectory:} /home/daimis.hechevarria
	\item {\bf uid:} daimis.hechevarria
\end{enumerate}

Después de generado el archivo LDIF, al igual que en módulo anterior este es utilizado para poblar o modificar el servidor LDAP utilizando un parser de LDIF que brinda la biblioteca python-ldap.

\section{Servidor MS SQL Server}

En la sección anterior mencionamos que el módulo que se encarga de generar el LDIF de los trabajadores y poblar el servidor LDAP requiere de la restauración de la salva de la base de datos de ASSETS en un servidor MS SQL Server. Este servidor se despliega como parte de la solución utilizando la herramienta Docker, las configuraciones se obtienen de las variables de entorno que se declaran en el archivo .env que se encuentra dentro de la raíz de la solución, el cual utiliza el archivo docker-compose.yml para mediante el uso de docker-compose automatizar el despliegue de este servicio.

\section{Implementaci\'on de la API}

La API se desarrolló utlizando el framework Flask de Python, debido a la eficiencia de este framework minimalista, y al poder de este lenguaje de programación en cuanto a soluciones ya implementadas (librerías). Consiste en un conjunto de recursos creados utilizando la librería flask-restful a los cuales se accede utilizando el protocolo HTTP (los recursos se publican como URLs).  Utilizando la librería python-ldap, el API se conecta y se comunica con el servidor LDAP. 

\subsection{Listado de recursos}

\subsubsection{/login}

\textbf{Método:} POST

\textbf{Argumentos:} [email, password]

\textbf{Objeto respuesta:} []

Este recurso es utilizado para autenticar a los usuarios en la API. Las credenciales de los usuarios son obtenidas del servidor LDAP (la autenticación es contra este servidor). Para esto utiliza python-ldap para buscar el usuario utilizando como argumento el correo electr\'onico, y un m\'etodo auxiliar para verificar la contraseña del mismo. En el caso de que las credenciales sean verificadas correctamente, utilizando el m\'odulo flask\_jwt\_extended, se genera y se almacena en una cookie del navegador un token jwt que puede utilizarse para realizar el resto de las peticiones como usuario ya autenticado.


\subsubsection{/logout}

\textbf{Método:} POST

\textbf{Argumentos:} []

\textbf{Objeto respuesta:} []

Este recurso es para terminar la sesión de un usuario autenticado en la API. Utiliza el m\'etodo unset\_jwt\_cookies del m\'odulo flask\_jwt\_extended para eliminar la cookie del navegador, y as\'i terminar la sesi\'on del usuario. 


\subsubsection{/usuarios}

\textbf{Método:} GET

\textbf{Argumentos:} [filtros]

\textbf{Objeto respuesta:} [{usuarios: [...]}]

Este recurso es para obtener la información pertinente a un conjunto de personas pertenecientes a la universidad en base a parámetros de búsqueda. Utiliza python-ldap para consultar el servidor LDAP y devuelve los resultados como JSON.

\subsubsection{/trabajadores}

\textbf{Métodos:} GET, POST, PATCH

\textbf{Argumentos:} GET: [filtros], POST: [ci], PATCH: []

\textbf{Objeto respuesta:} GET: [{'workers': [...]}], POST: [], PATCH: []

Este recurso es accesible mediante 3 métodos HTTP, para realizar 3 acciones distintas. Método GET para obtener la información pertinente a un conjunto de trabajadores pertenecientes a la universidad en base a parámetros de búsqueda, utilizando python-ldap para consultar el servidor LDAP, y devolviendo los resultados de la b\'usqueda como JSON. Método POST para agregar un trabajador al directorio a partir de su carnet de identidad. Este m\'etodo utiliza python-ldap para buscar si el trabajador existe en el directorio (o sea, si existe en las base de datos de Recursos Humanos de la Universidad), si no existe devuelve un mensaje de error, si existe, le crea un nombre de usuario y un correo en dependencia del \'area del trabajador utilizando el m\'etodo auxiliar \_\_generate\_new\_email\_\_. Por \'ultimo, el método PATCH se utiliza para actualizar los trabajadores en el directorio a partir de las Bases de Datos de Recursos Humanos (cuando se hace mención al directorio nos referimos al servidor LDAP que almacena la información), utilizando el m\'odulo de extracción de información ldif-from-database.

\subsubsection{/estudiantes}

\textbf{Métodos:} GET, POST, PATCH

\textbf{Argumentos:} GET: [filtros], POST: [ci], PATCH: []

\textbf{Objeto respuesta:} GET: [{'students': [...]}], POST: [], PATCH: []

Este recurso es accesible también mediante 3 métodos HTTP, para realizar 3 acciones distintas. Método GET para obtener la información pertinente a un conjunto de estudiantes pertenecientes a la universidad en base a parámetros de búsqueda, utilizando python-ldap para consultar el servidor LDAP, y devolviendo los resultados de la b\'usqueda como JSON. Método POST para agregar un estudiante al directorio a partir de su carnet de identidad. Este m\'etodo utiliza python-ldap para buscar si el estudiante existe en el directorio (o sea, si existe en sistema SIGENU), si no existe devuelve un mensaje de error, si existe, le crea un nombre de usuario y un correo en dependencia de la facultad utilizando el m\'etodo auxiliar \_\_generate\_new\_email\_\_. Por \'ultimo, el método PATCH se utiliza para actualizar los estudiantes en el directorio a partir de SIGENU (cuando se hace mención al directorio nos referimos al servidor LDAP que almacena la información), utilizando el m\'odulo de extracción de información sigenu-client.

\subsubsection{/externos}

\textbf{Métodos:} GET, POST

\textbf{Argumentos:} GET: [filtros], POST: [nombre, ci, cuota, ...]

\textbf{Objeto respuesta:} GET: [{'externs': [...]}], POST: []

Este recurso es accesible mediante 2 métodos HTTP, para realizar 2 acciones distintas. Método GET para obtener la información pertinente a un conjunto de externos de la universidad en base a parámetros de búsqueda, utilizando python-ldap para consultar el servidor LDAP, y devolviendo los resultados de la b\'usqueda como JSON. Método POST para agregar un externo (persona que utiliza de forma temporal los servicios de la universidad, como un profesor de otra universidad que viene a impartir un curso de verano, etc...) al directorio (cuando se hace mención al directorio nos referimos al servidor LDAP que almacena la información).


\subsubsection{/p/preguntasdeseguridad}

\textbf{Métodos:} GET, POST, PATCH, PUT

\textbf{Argumentos:} GET: [email], POST: [email, password, answers], PATCH: [email, password], PUT: [email, password, questions, answers]

\textbf{Objeto respuesta:} GET: [{'preguntas': [...]}], POST: [], PATCH: [{'preguntas': [...], 'respuestas': [...]}], PUT: []

Este recurso es accesible también mediante 4 métodos HTTP, para realizar 4 acciones distintas.  Método PATCH para obtener las preguntas y respuestas de seguridad de un usuario a verificando sus credenciales. Método POST para recuperar la contraseña a partir de las preguntas de seguridad. Método GET para obtener las preguntas de seguridad de un usuario a partir de su correo electr\'onico sin la necesidad de verificar la contraseña. M\'etodo PUT para cambiar las preguntas y respuestas de seguridad verificando las credenciales del usuario. Para la modificaci\'on y la consulta de los datos del servidor LDAP se utiliza python-ldap, y el m\'etodo auxiliar verify\_user\_password para verificar ls credenciales del usuario cuando es necesario.

\subsubsection{/p/cambiar}

\textbf{Métodos:} POST

\textbf{Argumentos:} POST: [email, oldpassword, newpassword]

\textbf{Objeto respuesta:} POST: []

Este recurso es para que un usuario del directorio pueda cambiar su contraseña. Verifica las credenciales actuales del usuario utilizando el m\'etodo auxiliar verify\_user\_password, y python-ldap para modificar la contraseña si las credenciales son correctas.

\subsubsection{/administradores}

\textbf{Métodos:} GET, PUT, DELETE

\textbf{Argumentos:} GET: [], PUT: [email], DELETE: [email]

\textbf{Objeto respuesta:} GET: [{'administradores': [...]}], PUT: [], DELETE: []

Este recurso es el encargado de gestionar la creación y eliminación de los administradores de la API. Es accesible mediante 3 métodos HTTP, para realizar 3 acciones distintas.  Método GET para obtener la lista de los administradores actuales de la API. Este método es utilizado por la interfaz para listar los administradores y poder gestionarlos. Método PUT para convertir un trabajador con una cuenta ya creada en el directorio en administrador. Utiliza el correo para identificar esta cuenta. Método DELETE para eliminar el rol de administrador de un usuario a partir de su correo. 

\section{Interfaz web}

La interfaz web se implementó como una aplicación independiente utilizando React como tecnología. Brinda las funcionalidades esenciales que actualmente presentan las interfaces del sitio de directorio, el sitio de cuentas y el sitio de gestión: la recuperación de contraseñas a partir de las preguntas de seguridad de un usuario; el cambio de contraseñas; la creación de cuentas tanto de trabajadores como de estudiantes a partir del carnet de identidad; la adición de externos; la consulta de información relacionada con usuarios.

\section{Proxy inverso}

A la hora de desplegar los servicios fundamentales que componen la solución implementada (dígase API e interfaz web), también se despliega mediante el uso de la herramienta Docker un proxy inverso que se encarga de redireccionar las peticiones hacia estos servicios de forma adecuada. Este proxy inverso se despliega utilizando una imagen de Docker previamente creada (y publicada en el repositorio hub.docker.com) de Nginx. Para configurar este servidor que hace la función de proxy inverso solo es necesario modificar el archivo nginx.conf en la raíz de la solución, ya que mediante el uso de Docker y docker-compose, al desplegar el servicio estas configuraciones son cargadas en el contenedor de Docker de manera adecuada y automática.



\section{Bolsa de servicios}

La bolsa de servicios no es más que una extensión de las funcionalidades básicas del API, para satisfacer las necesidades de los sistemas que necesitan consumir información que se obtiene a partir de los datos almacenados en el servidor LDAP, pero que no necesariamente están almacenados en el mismo, como la cuota de internet.

Por convención la url con la que se publican estos servicios contiene el prefijo /servicios/, por ejemplo, para obtener la cuota de internet de un estudiante se utiliza la url /servicios/cuotadeinternet/estudiantes, la cual recibe el correo del estudiante como parámetro.




  
