\begin{recomendations}

Se recomienda para continuar este trabajo, estudiar la posibilidad de automatizar la obtención de las bases de datos de ASSETS de las demás unidades presupuestadas que pertenecen a la Universidad de La Habana (en esta tesis se utilizó solamente la de la sede principal). Después de esta automatización, queda pendiente analizar la estructura particular de cada una de estas bases de datos para genralizar el módulo de extracción de información, permitiendo que genere el archivo LDIF incluyendo estas fuentes. Estas bases de datos faltantes son la del Jardín Botánico, la da IFAL, la de la UPA, la del ISDI y la del INSTEC.

También se recomienda completar la correspondencia entre las áreas de los trabajadores que se extraen de la base de datos de ASSETS que se utiliza y los dominios de correo. Actualmente esta correspondencia no está completa para algunos dominios, debido a lo ambiguas que son algunas de estas áreas, como por ejemplo Secretaría de Facultad, a veces no siendo específico a qué facultad hace referencia.

Se recomienda además terminar de incluir las funcionalidades de los sitios de gestión, cuentas y directorio, que le faltan a la interfaz implementada en esta solución (con su correspondiente complemento en la API). Esta tesis busca como objetivo a largo plazo centralizar todas estas interfaces independientes en una sola. Funcionalidades como el filtrado por semejanza de cadenas de texto aún no están implementados en la nueva interfaz.

\end{recomendations} 